\documentclass[11pt,show 
notes,notheorems,noamsthm,blank]{beamer} % default is 11pt
\usetheme{metropolis}
% \usepackage{beamerthemeshadow}
% \usepackage{times}
\usepackage[utf8]{inputenc}
\usepackage[scaled]{helvet} % ss
\usepackage[T1]{fontenc}
\usepackage{eurosym}
\usepackage{graphics, graphicx}
\usepackage{url}
% \usepackage{multirow}
\usepackage{listings}
\usepackage{epsfig}
%  \usepackage[portuguese]{babel}
%  \usepackage[latin1]{inputenc}  
\usepackage{fancyvrb}
\usepackage[prefix=tikzsym]{tikzsymbols}

\usepackage{fontawesome}


% \usepackage[marvosym]{tikzsymbols}
% \usepackage{rotating}
% \usepackage{subfigure}
% \usepackage{default}
 \usepackage{algorithmic}
 \usepackage{algorithm}
\usepackage{rotating}
\usepackage{hyperref}
\definecolor{links}{HTML}{FF5050}
\definecolor{inlinks}{HTML}{4D94FF}
\hypersetup{colorlinks,linkcolor=inlinks,urlcolor=links}
\beamertemplatetransparentcovered
%  \usepackage[portuguese]{babel}
%  \usepackage[latin1]{inputenc}  
\usepackage{rotating}
\usepackage{colortbl}
% \usepackage{moreverb}
% \usepackage{tikz}
% \usepackage{wrapfig}
\setbeamertemplate{bibliography item}{\insertbiblabel}
% \usepackage{tikz}
% \usetikzlibrary{arrows,positioning,shapes}
% \usetikzlibrary{shapes,calc,shadows}
\pgfdeclarelayer{background}
\pgfdeclarelayer{foreground}
% \pagestyle{fancy}
\pgfsetlayers{background,main,foreground}
\usepackage{url}
\usepackage{hyperref}
% \usepackage{color}X
\setbeamertemplate{navigation symbols}{}
% \renewcommand{\algorithmicrequire}{\textbf{Input:}}
% % \renewcommand{\algorithmicensure}{\textbf{Output:}}
% \hypersetup{
%     colorlinks,%
%     citecolor=BrickRed,%
%     filecolor=black,%
%     linkcolor=blue,%
%     urlcolor=BrickRed
% }





\usepackage{tikz}
\usetikzlibrary{calc,shapes.symbols,shapes.geometric,positioning,arrows,chains}
\usetikzlibrary{backgrounds,decorations,decorations.text,
decorations.pathreplacing}
% tiks definitions
\usetikzlibrary{arrows.meta}
\tikzstyle{default_rectangle} = [black, fill=white]
\tikzset{>=stealth}
\tikzset{every picture/.append style={font=\small}}
\usetikzlibrary{arrows, matrix, shadows}
\tikzset{>=stealth}
\tikzstyle{every picture}=[line width=0.75pt]
\tikzstyle{every node}=[font=\small]
\tikzstyle{default_rectangle} = [black, fill=white]
\definecolor{myblue}{RGB}{68,119,170}
\definecolor{myred}{RGB}{204,102,119}
\definecolor{mygreen}{RGB}{17,119,51}
\definecolor{myyellow}{RGB}{221,204,119}
% Definitions for figures
\tikzstyle{server}=[circle,draw,minimum size=2em,scale=.9,solid,fill=green!20]
\tikzstyle{site}=[circle,draw,minimum size=2em,scale=.9,solid,fill=red!20]
\tikzstyle{root}=[circle,draw,minimum size=2em,scale=.9,solid,fill=red!20]
\tikzstyle{user}=[circle,draw,minimum size=2em,scale=.9,solid,fill=yellow!20]
\tikzstyle{cache}=[rectangle,draw,minimum 
size=3em,scale=.9,solid,fill=blue!20,minimum height=1em]

% Definitions for figures
\tikzstyle{server}=[circle,draw,minimum size=2em,scale=.9,solid,fill=green!20]
\tikzstyle{site}=[circle,draw,minimum size=2em,scale=.9,solid,fill=red!20]
\tikzstyle{root}=[circle,draw,minimum size=2em,scale=.9,solid,fill=red!20]
\tikzstyle{user}=[circle,draw,minimum size=2em,scale=.9,solid,fill=yellow!20]
\tikzstyle{cache}=[rectangle,draw,minimum 
size=3em,scale=.9,solid,fill=blue!20,minimum height=1em]


\usepackage{pgf}
\pgfdeclareimage[height=0.5cm]{logo}{fig/sidn-delft}
% \addtobeamertemplate{navigation symbols}{}{%
%     \usebeamerfont{footline}%
%     \usebeamercolor[fg]{footline}%
%     \hspace{1em}%
%     \insertframenumber/\inserttotalframenumber
% }

% \setbeamercolor{footline}{fg=blue}
\setbeamerfont{footline}{series=\bfseries}


% \setbeamerfont{}{family=\rmfamily,series=\bfseries,size={\fontsize{12}{36}}}

%  \logo{\pgfuseimage{logo}}

\begin{document}
\title{draft-moura-dnsop-authoritative-recommendations-02}  
\author[Moura et. al]{\textbf{Giovane C. M. Moura}$^{1,2}$, Wes Hardaker$^3$, 
\\John Heidemann$^3$, Marco Davids$^1$\\}
\vspace{-0.3cm}
\institute{$^1$SIDN Labs, $^2$TU Delft, $^3$USC/ISI\\
%    \includegraphics[width=9cm]{fig/logos.png}
}


   
   
\date {IETF 104 -- Prague, CZ\\
2019-03-XX\\


}  

\frame{\titlepage} 






\begin{frame}
\frametitle{Draft History}


\begin{itemize}

\item \textbf{Today:} first time presented at DNSOP 

\item Versions and mailing list discussion:

\begin{itemize}
  \item \textbf{00 (2018-11-28):}   
\href{https://mailarchive.ietf.org/arch/msg/dnsop/AMMr6dDDUmShnG90URv6AJCY_VQ}{
 list thread}

  \item \textbf{01 (2018-12-20):} 
\href{https://mailarchive.ietf.org/arch/msg/dnsop/2R8Ab4-7sKmOY7-XcJ3yLSq6Gcc}{
 list thread (no responses)} 


  \item \textbf{02 (2019-03-09):}   
\href{https://mailarchive.ietf.org/arch/msg/dnsop/}{
 list thread }
\end{itemize}

\item Github link: \small
\url{https://github.com/gmmoura/draft-moura-dnsop-authoritative-recommendations}



\end{itemize}


\end{frame}


\begin{frame}
 \frametitle{Context}
 
 \begin{itemize}

  \item We are a group of 13 people that have had 5 relevant research papers 
for large Auth Server OPs:
  \begin{itemize}
   \item \footnotesize Draft authors + Ricardo de O Schmidt, Wouter B. de 
Vries, Moritz M\"{u}ller, Lan Wei,  Cristian Hesselman, Jan Harm Kuipers, 
Pieter-Tjerk de Boer and Aiko  Pras.
  \end{itemize}

 \item These papers have \textit{recommendations} backed by 
large-scale, Internet-wide measurements presented at conferences:
\begin{itemize}
 \item 4x ACM IMC
 \item 1x PAM
\end{itemize}


\item However, papers tend to be \textit{long}, \textit{detailed} -- they 
explain \textit{why} 

 \end{itemize}

\end{frame}



\begin{frame}[fragile]
 \frametitle{Context}
 
 
\textbf{This draft:}



 \lstset{language=Python}
 \lstset{frame=lines}
%  \lstset{caption={Insert code directly in your document}}
 \lstset{label={lst:code_direct}}
  \lstset{basicstyle=\footnotesize}
  
  
\begin{lstlisting}
papers=[]
papers.append(Moura16b)
papers.append(Mueller17b)
papers.append(Schmidt17a)
papers.append(Vries17b)
papers.append(Moura18b)

for p in papers:
  recommendation = TLDR(p)
  print(recommendation)
\end{lstlisting}

\begin{itemize}
 \item Tangile,direct language to OPs folks interested on \textit{what} to do
 \item Reader is referred to papers to understand \textit{why}
\end{itemize}


\end{frame}

 
\begin{frame}
 \frametitle{Questions?}
 
 
 \begin{itemize}
  \item Draft on GitHub: \small
\url{https://github.com/gmmoura/draft-moura-dnsop-authoritative-recommendations}
 \end{itemize}

\end{frame}


 
 \begin{frame}[allowframebreaks] {References}
 
\bibliographystyle{plain}
% \balance
\small
\bibliography{../paper,../rfc}
\end{frame}
 
\end{document}
